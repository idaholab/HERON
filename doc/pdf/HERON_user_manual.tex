% \input{} or \include{} lines will added later
\documentclass[pdf,12pt]{../INLreport}
\usepackage{times}
\usepackage{longtable}
\usepackage[FIGBOTCAP,normal,bf,tight]{subfigure}
\usepackage{amsmath}
\usepackage{amssymb}
\usepackage[labelfont=bf]{caption}
\usepackage{pifont}
\usepackage{enumerate}
\usepackage{listings}
\usepackage{fullpage}
\usepackage{xcolor}          % Using xcolor for more robust color specification
\usepackage{ifthen}          % For simple checking in newcommand blocks
\usepackage{textcomp}
%\usepackage{authblk}         % For making the author list look prettier
%\renewcommand\Authsep{,~\,}
% Custom colors
\definecolor{deepblue}{rgb}{0,0,0.5}
\definecolor{deepred}{rgb}{0.6,0,0}
\definecolor{deepgreen}{rgb}{0,0.5,0}
\definecolor{forestgreen}{RGB}{34,139,34}
\definecolor{orangered}{RGB}{239,134,64}
\definecolor{darkblue}{rgb}{0.0,0.0,0.6}
\definecolor{gray}{rgb}{0.4,0.4,0.4}
\lstset {
  basicstyle=\ttfamily,
  frame=single
}
\setcounter{secnumdepth}{5}
\lstdefinestyle{XML} {
    language=XML,
    extendedchars=true,
    breaklines=true,
    breakatwhitespace=true,
%    emph={name,dim,interactive,overwrite},
    emphstyle=\color{red},
    basicstyle=\ttfamily,
%    columns=fullflexible,
    commentstyle=\color{gray}\upshape,
    morestring=[b]",
    morecomment=[s]{<?}{?>},
    morecomment=[s][\color{forestgreen}]{<!--}{-->},
    keywordstyle=\color{cyan},
    stringstyle=\ttfamily\color{black},
    tagstyle=\color{darkblue}\bf\ttfamily,
    morekeywords={name,type},
%    morekeywords={name,attribute,source,variables,version,type,release,x,z,y,xlabel,ylabel,how,text,param1,param2,color,label},
}
\lstset{language=python,upquote=true}
\usepackage{titlesec}
\newcommand{\sectionbreak}{\clearpage}
\setcounter{secnumdepth}{4}
\usepackage[utf8]{inputenc}
% Default fixed font does not support bold face
\DeclareFixedFont{\ttb}{T1}{txtt}{bx}{n}{9} % for bold
\DeclareFixedFont{\ttm}{T1}{txtt}{m}{n}{9}  % for normal
\usepackage{listings}
\newcommand\pythonstyle{\lstset{
language=Python,
basicstyle=\ttm,
otherkeywords={self, none, return},             % Add keywords here
keywordstyle=\ttb\color{deepblue},
emph={MyClass,__init__},          % Custom highlighting
emphstyle=\ttb\color{deepred},    % Custom highlighting style
stringstyle=\color{deepgreen},
frame=tb,                         % Any extra options here
showstringspaces=false            %
}}
% Python environment
\lstnewenvironment{python}[1][]
{
\pythonstyle
\lstset{#1}
}
{}
% Python for external files
\newcommand\pythonexternal[2][]{{
\pythonstyle
\lstinputlisting[#1]{#2}}}
\lstnewenvironment{xml}
{}
{}
% Python for inline
\newcommand\pythoninline[1]{{\pythonstyle\lstinline!#1!}}
% Named Colors for the comments below (Attempted to match git symbol colors)
\definecolor{RScolor}{HTML}{8EB361}  % Sonat (adjusted for clarity)
\definecolor{DPMcolor}{HTML}{E28B8D} % Dan
\definecolor{JCcolor}{HTML}{82A8D9}  % Josh (adjusted for clarity)
\definecolor{AAcolor}{HTML}{8D7F44}  % Andrea
\definecolor{CRcolor}{HTML}{AC39CE}  % Cristian
\definecolor{RKcolor}{HTML}{3ECC8D}  % Bob (adjusted for clarity)
\definecolor{DMcolor}{HTML}{276605}  % Diego (adjusted for clarity)
\definecolor{PTcolor}{HTML}{990000}  % Paul
\def\DRAFT{} % Uncomment this if you want to see the notes people have been adding
% Comment command for developers (Should only be used under active development)
\ifdefined\DRAFT
  \newcommand{\nameLabeler}[3]{\textcolor{#2}{[[#1: #3]]}}
\else
  \newcommand{\nameLabeler}[3]{}
\fi
\newcommand{\alfoa}[1] {\nameLabeler{Andrea}{AAcolor}{#1}}
\newcommand{\cristr}[1] {\nameLabeler{Cristian}{CRcolor}{#1}}
\newcommand{\talbpaul}[1] {\nameLabeler{Paul}{PTcolor}{#1}}
% Commands for making the LaTeX a bit more uniform and cleaner
\newcommand{\TODO}[1]    {\textcolor{red}{\textit{(#1)}}}
\newcommand{\xmlAttrRequired}[1] {\textcolor{red}{\textbf{\texttt{#1}}}}
\newcommand{\xmlAttr}[1] {\textcolor{cyan}{\textbf{\texttt{#1}}}}
\newcommand{\xmlNodeRequired}[1] {\textcolor{deepblue}{\textbf{\texttt{<#1>}}}}
\newcommand{\xmlNode}[1] {\textcolor{darkblue}{\textbf{\texttt{<#1>}}}}
\newcommand{\xmlString}[1] {\textcolor{black}{\textbf{\texttt{'#1'}}}}
\newcommand{\xmlDesc}[1] {\textbf{\textit{#1}}} % Maybe a misnomer, but I am
                                                % using this to detail the data
                                                % type and necessity of an XML
                                                % node or attribute,
                                                % xmlDesc = XML description
\newcommand{\default}[1]{~\\*\textit{Default: #1}}
\newcommand{\nb} {\textcolor{deepgreen}{\textbf{~Note:}}~}
\usepackage{bm}
\newcommand{\tensor}[1]{{\bm{#1}}}
\renewcommand{\vec}{\bm}
\newcommand{\unit}[1]{\hat{\bm{#1}}}
\newcommand{\scalarunit}[1]{\hat{#1}}
\usepackage{booktabs}
\usepackage{stmaryrd}
\usepackage{hyperref}
\hypersetup{
    colorlinks,
    citecolor=black,
    filecolor=black,
    linkcolor=black,
    urlcolor=black
}
\newcommand{\wiki}{\href{https://github.com/idaholab/raven/wiki}{RAVEN wiki}}
\usepackage{cite}
\raggedbottom
\setcounter{secnumdepth}{5} % show 5 levels of subsection
\setcounter{tocdepth}{5} % include 5 levels of subsection in table of contents
\title{HERON User Manual}
\author{
\textbf{\textit{Project Manager:}}
 \\Cristian Rabiti\\
 \textbf{\textit{Principal Investigator and Technical Leader:}}
\\Paul W. Talbot\\
\textbf{\textit{Main Developers:}}
\\Paul W. Talbot\\
Abhinav Gairola\\
\textbf{\textit{Additional  Developers:}}
\\Jia Zhou
}
\date{}
\SANDnum{INL/EXT-01-00002
}}}}}}}}}}}}}}
\SANDprintDate{\today}
\SANDauthor{Cristian Rabiti, Paul W. Talbot, Abhinav Gairola, Jia Zhou}
\SANDreleaseType{Revision 1}
\def\component#1{\texttt{#1}}
% ---------------------------------------------------------------------------- %
\newcommand{\systemtau}{\tensor{\tau}_{\!\text{SUPG}}}
% Added by Sonat
\usepackage{placeins}
\usepackage{array}
\newcolumntype{L}[1]{>{\raggedright\let\newline\\\arraybackslash\hspace{0pt}}m{#1}}
\newcolumntype{C}[1]{>{\centering\let\newline\\\arraybackslash\hspace{0pt}}m{#1}}
\newcolumntype{R}[1]{>{\raggedleft\let\newline\\\arraybackslash\hspace{0pt}}m{#1}}
 \begin{document}
    \maketitle
    \SANDmain
    \tableofcontents
\section{Introduction}
HERON is a generic software plugin of RAVEN to perform stochastic technoeconomic analysis of grid energy-resource systems with economic drivers. The development targets analysis of electricity and
secondary product generation and consumption in regional balancing areas, including flexibility to include arbitrary resources as well as arbitrary resource consumers and producers. HERON is developed to drive optimization via economic drivers such as system cost minimization, profitability, and net present value (NPV) maximization. As a plugin of RAVEN, HERON provides two primary functions: the automatic generation of RAVEN workflows, and models for optimizing high-resolution dispatch of arbitrary systems including resources, resource consumers, and resource producers. HERON leverages the synthetic history training and generation tools, sampling workflows, code Application Programming Interfaces (API), and optimization schemes. 
\section{ Installation and how to run}
\subsection{Installation}
As a plugin of RAVEN, HERON is installed as a submodule. RAVEN maintains up-to-date
instructions for plugin installation in its manuals and other documentation. As of this writing, plugins
are installed from the command line from within the top level of the RAVEN repository as
\begin{verbatim}
scripts/install_plugins.py ‐s heron
\end{verbatim}
Note that RAVEN must be installed completely to use HERON and its components. At this writing
HERON introduces one additional Python library to the standard set of RAVEN \texttt{Python} dependencies:
\texttt{dill}. The serialization package \texttt{dill} enables information transfer between the HERON templating
algorithm and dispatching algorithm not possible with the standard library set. This library, along with
any future additional libraries required, should be installed correctly via RAVEN’s documented
installation procedure for including plugins.
HERON also requires use of the RAVEN plugin\textt{tCashFlow} , a utility for economic analysis.
As of this writing, \texttt{CashFlow} is an officially supported plugin for RAVEN and is installed standard when plugins are included in RAVEN installation. 
\subsection{How to run}
The code can be run easily on a \texttt{linux/mac} system by defining an alias to the \texttt{Python} script `\texttt{main.py}' in the \texttt{src} folder of  HERON.

\section{Cases Introduction}HERON relies on this \xmlNode{xml} node which informs the algorithm as to how the case has to be processed by using the predefined metrics described in the following sections.




\subsection{Case}
  The \xmlNode{Case} contains    the basic parameters needed for a HERON case.

  The \xmlNode{Case} node recognizes the following parameters:
    \begin{itemize}
      \item \xmlAttr{name}: \xmlDesc{string, required}, 
        An appropriate user defined name of the case.
  \end{itemize}

  The \xmlNode{Case} node recognizes the following subnodes:
  \begin{itemize}
    \item \xmlNode{mode}: \xmlDesc{[min, max, sweep]}, 
      Minimize, maximize or sweep over multiple values of capacities.

    \item \xmlNode{metric}: \xmlDesc{[NPV, lcoe]}, 
      This metric can be NPV (Net Present Value) and lcoe (levelized cost of energy) used for
      techno-economic analysis of the power plants.

    \item \xmlNode{differential}: \xmlDesc{[yes, y, true, t, si, vero, dajie, oui, ja, yao, verum, evet, dogru, 1, on, no, n, false, f, nono, falso, nahh, non, nicht, bu, falsus, hayir, yanlis, 0, off, Yes, Y, True, T, Si, Vero, Dajie, Oui, Ja, Yao, Verum, Evet, Dogru, 1, On, No, N, False, F, Nono, Falso, Nahh, Non, Nicht, Bu, Falsus, Hayir, Yanlis, 0, Off]}, 
      Differential represents the additional cashflow generated when building additional capacities.
      This value can be either \xmlString{True} or \xmlString{False}.

    \item \xmlNode{num\_arma\_samples}: \xmlDesc{integer}, 
      Number of copies of the trained signals.

    \item \xmlNode{timestep\_interval}: \xmlDesc{integer}, 
      Time step interval between two values of signal.

    \item \xmlNode{history\_length}: \xmlDesc{integer}, 
      Total length of one realization of the ARMA signal.

    \item \xmlNode{economics}:
      \xmlNode{economics} contains the details of the econometrics     computations to be performed
      by the code.

      The \xmlNode{economics} node recognizes the following subnodes:
      \begin{itemize}
        \item \xmlNode{ProjectTime}: \xmlDesc{float}, 
          Total length of the project.

        \item \xmlNode{DiscountRate}: \xmlDesc{float}, 
          Interest rate required to compute the discounted cashflow (DCF)

        \item \xmlNode{tax}: \xmlDesc{float}, 
          Taxation rate is a metric which represents the      rate at which an individual or
          corporation is taxed.

        \item \xmlNode{inflation}: \xmlDesc{float}, 
          Inflation rate is a metric which represents the     the rate at which the average price
          level of a basket of selected goods and services in an economy increases over some period
          of time.

        \item \xmlNode{verbosity}: \xmlDesc{integer}, 
          Length of the output argument.
      \end{itemize}

    \item \xmlNode{dispatch\_increment}: \xmlDesc{float}, 
      This is the amount of resource to be dispatched in a fixed time interval.
      The \xmlNode{dispatch\\_increment} node recognizes the following parameters:
        \begin{itemize}
          \item \xmlAttr{resource}: \xmlDesc{string, required}, 
            Resource to be consumed or produced.
      \end{itemize}
  \end{itemize}

\section{Economics Introduction}The \xmlNode{Economics} node describes the basic metrics used to compute the key economic parameters for the techno-economic analysis of component configurations.




\subsection{CashFlow}
  -- no description yet --

  The \xmlNode{CashFlow} node recognizes the following parameters:
    \begin{itemize}
      \item \xmlAttr{name}: \xmlDesc{string, required}, 
        -- no description yet --
      \item \xmlAttr{type}: \xmlDesc{string, required}, 
        -- no description yet --
      \item \xmlAttr{taxable}: \xmlDesc{bool, required}, 
        -- no description yet --
      \item \xmlAttr{inflation}: \xmlDesc{string, required}, 
        -- no description yet --
      \item \xmlAttr{mult\\_target}: \xmlDesc{bool, required}, 
        -- no description yet --
      \item \xmlAttr{period}: \xmlDesc{period\_opts, optional}, 
        -- no description yet --
  \end{itemize}

  The \xmlNode{CashFlow} node recognizes the following subnodes:
  \begin{itemize}
    \item \xmlNode{driver}:
      The node \xmlNode{driver} has the following \textcolor{red}{\textit{ValuedParams}} options:

      The \xmlNode{driver} node recognizes the following subnodes:
      \begin{itemize}
        \item \xmlNode{fixed\_value}: \xmlDesc{float}, 
          It can be a fixed value.

        \item \xmlNode{sweep\_values}: \xmlDesc{comma-separated floats}, 
          It can be a value which is to be swept over multiple values.

        \item \xmlNode{opt\_bounds}: \xmlDesc{comma-separated floats}, 
          Opt bounds.

        \item \xmlNode{ARMA}: \xmlDesc{string}, 
          It can be an \textbf{A}uto \textbf{R}egressive \textbf{M}oving \textbf{A}verage value.
          The \xmlNode{ARMA} node recognizes the following parameters:
            \begin{itemize}
              \item \xmlAttr{variable}: \xmlDesc{string, optional}, 
                Variable generated by an ARMA model.
          \end{itemize}

        \item \xmlNode{Function}: \xmlDesc{string}, 
          It can be a value generated by running a function.
          The \xmlNode{Function} node recognizes the following parameters:
            \begin{itemize}
              \item \xmlAttr{method}: \xmlDesc{string, optional}, 
                The method containing the function.
          \end{itemize}

        \item \xmlNode{variable}: \xmlDesc{string}, 
          Variable

        \item \xmlNode{growth}: \xmlDesc{float}, 
          Growth factor required to grow the variable from one year to another.
          The \xmlNode{growth} node recognizes the following parameters:
            \begin{itemize}
              \item \xmlAttr{mode}: \xmlDesc{growthType, optional}, 
                The growth mode can be linear or exponential.
          \end{itemize}
      \end{itemize}

    \item \xmlNode{reference\_price}:
      The node \xmlNode{reference\_price} has the following \textcolor{red}{\textit{ValuedParams}}
      options:

      The \xmlNode{reference\_price} node recognizes the following subnodes:
      \begin{itemize}
        \item \xmlNode{fixed\_value}: \xmlDesc{float}, 
          It can be a fixed value.

        \item \xmlNode{sweep\_values}: \xmlDesc{comma-separated floats}, 
          It can be a value which is to be swept over multiple values.

        \item \xmlNode{opt\_bounds}: \xmlDesc{comma-separated floats}, 
          Opt bounds.

        \item \xmlNode{ARMA}: \xmlDesc{string}, 
          It can be an \textbf{A}uto \textbf{R}egressive \textbf{M}oving \textbf{A}verage value.
          The \xmlNode{ARMA} node recognizes the following parameters:
            \begin{itemize}
              \item \xmlAttr{variable}: \xmlDesc{string, optional}, 
                Variable generated by an ARMA model.
          \end{itemize}

        \item \xmlNode{Function}: \xmlDesc{string}, 
          It can be a value generated by running a function.
          The \xmlNode{Function} node recognizes the following parameters:
            \begin{itemize}
              \item \xmlAttr{method}: \xmlDesc{string, optional}, 
                The method containing the function.
          \end{itemize}

        \item \xmlNode{variable}: \xmlDesc{string}, 
          Variable

        \item \xmlNode{growth}: \xmlDesc{float}, 
          Growth factor required to grow the variable from one year to another.
          The \xmlNode{growth} node recognizes the following parameters:
            \begin{itemize}
              \item \xmlAttr{mode}: \xmlDesc{growthType, optional}, 
                The growth mode can be linear or exponential.
          \end{itemize}
      \end{itemize}

    \item \xmlNode{reference\_driver}:
      The node \xmlNode{reference\_driver} has the following \textcolor{red}{\textit{ValuedParams}}
      options:

      The \xmlNode{reference\_driver} node recognizes the following subnodes:
      \begin{itemize}
        \item \xmlNode{fixed\_value}: \xmlDesc{float}, 
          It can be a fixed value.

        \item \xmlNode{sweep\_values}: \xmlDesc{comma-separated floats}, 
          It can be a value which is to be swept over multiple values.

        \item \xmlNode{opt\_bounds}: \xmlDesc{comma-separated floats}, 
          Opt bounds.

        \item \xmlNode{ARMA}: \xmlDesc{string}, 
          It can be an \textbf{A}uto \textbf{R}egressive \textbf{M}oving \textbf{A}verage value.
          The \xmlNode{ARMA} node recognizes the following parameters:
            \begin{itemize}
              \item \xmlAttr{variable}: \xmlDesc{string, optional}, 
                Variable generated by an ARMA model.
          \end{itemize}

        \item \xmlNode{Function}: \xmlDesc{string}, 
          It can be a value generated by running a function.
          The \xmlNode{Function} node recognizes the following parameters:
            \begin{itemize}
              \item \xmlAttr{method}: \xmlDesc{string, optional}, 
                The method containing the function.
          \end{itemize}

        \item \xmlNode{variable}: \xmlDesc{string}, 
          Variable

        \item \xmlNode{growth}: \xmlDesc{float}, 
          Growth factor required to grow the variable from one year to another.
          The \xmlNode{growth} node recognizes the following parameters:
            \begin{itemize}
              \item \xmlAttr{mode}: \xmlDesc{growthType, optional}, 
                The growth mode can be linear or exponential.
          \end{itemize}
      \end{itemize}

    \item \xmlNode{scaling\_factor\_x}:
      The node \xmlNode{scaling\_factor\_x} has the following \textcolor{red}{\textit{ValuedParams}}
      options:

      The \xmlNode{scaling\_factor\_x} node recognizes the following subnodes:
      \begin{itemize}
        \item \xmlNode{fixed\_value}: \xmlDesc{float}, 
          It can be a fixed value.

        \item \xmlNode{sweep\_values}: \xmlDesc{comma-separated floats}, 
          It can be a value which is to be swept over multiple values.

        \item \xmlNode{opt\_bounds}: \xmlDesc{comma-separated floats}, 
          Opt bounds.

        \item \xmlNode{ARMA}: \xmlDesc{string}, 
          It can be an \textbf{A}uto \textbf{R}egressive \textbf{M}oving \textbf{A}verage value.
          The \xmlNode{ARMA} node recognizes the following parameters:
            \begin{itemize}
              \item \xmlAttr{variable}: \xmlDesc{string, optional}, 
                Variable generated by an ARMA model.
          \end{itemize}

        \item \xmlNode{Function}: \xmlDesc{string}, 
          It can be a value generated by running a function.
          The \xmlNode{Function} node recognizes the following parameters:
            \begin{itemize}
              \item \xmlAttr{method}: \xmlDesc{string, optional}, 
                The method containing the function.
          \end{itemize}

        \item \xmlNode{variable}: \xmlDesc{string}, 
          Variable

        \item \xmlNode{growth}: \xmlDesc{float}, 
          Growth factor required to grow the variable from one year to another.
          The \xmlNode{growth} node recognizes the following parameters:
            \begin{itemize}
              \item \xmlAttr{mode}: \xmlDesc{growthType, optional}, 
                The growth mode can be linear or exponential.
          \end{itemize}
      \end{itemize}

    \item \xmlNode{depreciate}: \xmlDesc{integer}, 
      -- no description yet --
  \end{itemize}

\section{Components Introduction}Lorem ipsum dolor sit amet, consectetur adipiscing elit, sed do eiusmod tempor incididunt ut labore et dolore magna aliqua.



\subsection{Component}
  -- no description yet --

  The \xmlNode{Component} node recognizes the following parameters:
    \begin{itemize}
      \item \xmlAttr{name}: \xmlDesc{string, required}, 
        -- no description yet --
  \end{itemize}

  The \xmlNode{Component} node recognizes the following subnodes:
  \begin{itemize}
    \item \xmlNode{produces}:
      -- no description yet --
      The \xmlNode{produces} node recognizes the following parameters:
        \begin{itemize}
          \item \xmlAttr{resource}: \xmlDesc{string_list, required}, 
            -- no description yet --
          \item \xmlAttr{dispatch}: \xmlDesc{dispatch_opts, optional}, 
            -- no description yet --
      \end{itemize}

      The \xmlNode{produces} node recognizes the following subnodes:
      \begin{itemize}
        \item \xmlNode{capacity}:
          -- no description yet --
          The \xmlNode{capacity} node recognizes the following parameters:
            \begin{itemize}
              \item \xmlAttr{resource}: \xmlDesc{string, optional}, 
                -- no description yet --
          \end{itemize}

          The \xmlNode{capacity} node recognizes the following subnodes:
          \begin{itemize}
            \item \xmlNode{fixed_value}:\xmlDesc{float}, 
              -- no description yet --

            \item \xmlNode{sweep_values}:\xmlDesc{float_list}, 
              -- no description yet --

            \item \xmlNode{opt_bounds}:\xmlDesc{float_list}, 
              -- no description yet --

            \item \xmlNode{ARMA}:\xmlDesc{string}, 
              -- no description yet --
              The \xmlNode{ARMA} node recognizes the following parameters:
                \begin{itemize}
                  \item \xmlAttr{variable}: \xmlDesc{string, optional}, 
                    -- no description yet --
              \end{itemize}

            \item \xmlNode{Function}:\xmlDesc{string}, 
              -- no description yet --
              The \xmlNode{Function} node recognizes the following parameters:
                \begin{itemize}
                  \item \xmlAttr{method}: \xmlDesc{string, optional}, 
                    -- no description yet --
              \end{itemize}

            \item \xmlNode{variable}:\xmlDesc{string}, 
              -- no description yet --

            \item \xmlNode{growth}:\xmlDesc{float}, 
              -- no description yet --
              The \xmlNode{growth} node recognizes the following parameters:
                \begin{itemize}
                  \item \xmlAttr{mode}: \xmlDesc{growthType, optional}, 
                    -- no description yet --
              \end{itemize}
          \end{itemize}

        \item \xmlNode{minimum}:
          -- no description yet --
          The \xmlNode{minimum} node recognizes the following parameters:
            \begin{itemize}
              \item \xmlAttr{resource}: \xmlDesc{string, optional}, 
                -- no description yet --
          \end{itemize}

          The \xmlNode{minimum} node recognizes the following subnodes:
          \begin{itemize}
            \item \xmlNode{fixed_value}:\xmlDesc{float}, 
              -- no description yet --

            \item \xmlNode{sweep_values}:\xmlDesc{float_list}, 
              -- no description yet --

            \item \xmlNode{opt_bounds}:\xmlDesc{float_list}, 
              -- no description yet --

            \item \xmlNode{ARMA}:\xmlDesc{string}, 
              -- no description yet --
              The \xmlNode{ARMA} node recognizes the following parameters:
                \begin{itemize}
                  \item \xmlAttr{variable}: \xmlDesc{string, optional}, 
                    -- no description yet --
              \end{itemize}

            \item \xmlNode{Function}:\xmlDesc{string}, 
              -- no description yet --
              The \xmlNode{Function} node recognizes the following parameters:
                \begin{itemize}
                  \item \xmlAttr{method}: \xmlDesc{string, optional}, 
                    -- no description yet --
              \end{itemize}

            \item \xmlNode{variable}:\xmlDesc{string}, 
              -- no description yet --

            \item \xmlNode{growth}:\xmlDesc{float}, 
              -- no description yet --
              The \xmlNode{growth} node recognizes the following parameters:
                \begin{itemize}
                  \item \xmlAttr{mode}: \xmlDesc{growthType, optional}, 
                    -- no description yet --
              \end{itemize}
          \end{itemize}

        \item \xmlNode{consumes}:\xmlDesc{string_list}, 
          -- no description yet --

        \item \xmlNode{transfer}:
          -- no description yet --

          The \xmlNode{transfer} node recognizes the following subnodes:
          \begin{itemize}
            \item \xmlNode{fixed_value}:\xmlDesc{float}, 
              -- no description yet --

            \item \xmlNode{sweep_values}:\xmlDesc{float_list}, 
              -- no description yet --

            \item \xmlNode{opt_bounds}:\xmlDesc{float_list}, 
              -- no description yet --

            \item \xmlNode{ARMA}:\xmlDesc{string}, 
              -- no description yet --
              The \xmlNode{ARMA} node recognizes the following parameters:
                \begin{itemize}
                  \item \xmlAttr{variable}: \xmlDesc{string, optional}, 
                    -- no description yet --
              \end{itemize}

            \item \xmlNode{Function}:\xmlDesc{string}, 
              -- no description yet --
              The \xmlNode{Function} node recognizes the following parameters:
                \begin{itemize}
                  \item \xmlAttr{method}: \xmlDesc{string, optional}, 
                    -- no description yet --
              \end{itemize}

            \item \xmlNode{variable}:\xmlDesc{string}, 
              -- no description yet --

            \item \xmlNode{growth}:\xmlDesc{float}, 
              -- no description yet --
              The \xmlNode{growth} node recognizes the following parameters:
                \begin{itemize}
                  \item \xmlAttr{mode}: \xmlDesc{growthType, optional}, 
                    -- no description yet --
              \end{itemize}
          \end{itemize}
      \end{itemize}

    \item \xmlNode{stores}:
      -- no description yet --
      The \xmlNode{stores} node recognizes the following parameters:
        \begin{itemize}
          \item \xmlAttr{resource}: \xmlDesc{string_list, required}, 
            -- no description yet --
          \item \xmlAttr{dispatch}: \xmlDesc{dispatch_opts, optional}, 
            -- no description yet --
      \end{itemize}

      The \xmlNode{stores} node recognizes the following subnodes:
      \begin{itemize}
        \item \xmlNode{capacity}:
          -- no description yet --
          The \xmlNode{capacity} node recognizes the following parameters:
            \begin{itemize}
              \item \xmlAttr{resource}: \xmlDesc{string, optional}, 
                -- no description yet --
          \end{itemize}

          The \xmlNode{capacity} node recognizes the following subnodes:
          \begin{itemize}
            \item \xmlNode{fixed_value}:\xmlDesc{float}, 
              -- no description yet --

            \item \xmlNode{sweep_values}:\xmlDesc{float_list}, 
              -- no description yet --

            \item \xmlNode{opt_bounds}:\xmlDesc{float_list}, 
              -- no description yet --

            \item \xmlNode{ARMA}:\xmlDesc{string}, 
              -- no description yet --
              The \xmlNode{ARMA} node recognizes the following parameters:
                \begin{itemize}
                  \item \xmlAttr{variable}: \xmlDesc{string, optional}, 
                    -- no description yet --
              \end{itemize}

            \item \xmlNode{Function}:\xmlDesc{string}, 
              -- no description yet --
              The \xmlNode{Function} node recognizes the following parameters:
                \begin{itemize}
                  \item \xmlAttr{method}: \xmlDesc{string, optional}, 
                    -- no description yet --
              \end{itemize}

            \item \xmlNode{variable}:\xmlDesc{string}, 
              -- no description yet --

            \item \xmlNode{growth}:\xmlDesc{float}, 
              -- no description yet --
              The \xmlNode{growth} node recognizes the following parameters:
                \begin{itemize}
                  \item \xmlAttr{mode}: \xmlDesc{growthType, optional}, 
                    -- no description yet --
              \end{itemize}
          \end{itemize}

        \item \xmlNode{minimum}:
          -- no description yet --
          The \xmlNode{minimum} node recognizes the following parameters:
            \begin{itemize}
              \item \xmlAttr{resource}: \xmlDesc{string, optional}, 
                -- no description yet --
          \end{itemize}

          The \xmlNode{minimum} node recognizes the following subnodes:
          \begin{itemize}
            \item \xmlNode{fixed_value}:\xmlDesc{float}, 
              -- no description yet --

            \item \xmlNode{sweep_values}:\xmlDesc{float_list}, 
              -- no description yet --

            \item \xmlNode{opt_bounds}:\xmlDesc{float_list}, 
              -- no description yet --

            \item \xmlNode{ARMA}:\xmlDesc{string}, 
              -- no description yet --
              The \xmlNode{ARMA} node recognizes the following parameters:
                \begin{itemize}
                  \item \xmlAttr{variable}: \xmlDesc{string, optional}, 
                    -- no description yet --
              \end{itemize}

            \item \xmlNode{Function}:\xmlDesc{string}, 
              -- no description yet --
              The \xmlNode{Function} node recognizes the following parameters:
                \begin{itemize}
                  \item \xmlAttr{method}: \xmlDesc{string, optional}, 
                    -- no description yet --
              \end{itemize}

            \item \xmlNode{variable}:\xmlDesc{string}, 
              -- no description yet --

            \item \xmlNode{growth}:\xmlDesc{float}, 
              -- no description yet --
              The \xmlNode{growth} node recognizes the following parameters:
                \begin{itemize}
                  \item \xmlAttr{mode}: \xmlDesc{growthType, optional}, 
                    -- no description yet --
              \end{itemize}
          \end{itemize}

        \item \xmlNode{rate}:
          -- no description yet --

          The \xmlNode{rate} node recognizes the following subnodes:
          \begin{itemize}
            \item \xmlNode{fixed_value}:\xmlDesc{float}, 
              -- no description yet --

            \item \xmlNode{sweep_values}:\xmlDesc{float_list}, 
              -- no description yet --

            \item \xmlNode{opt_bounds}:\xmlDesc{float_list}, 
              -- no description yet --

            \item \xmlNode{ARMA}:\xmlDesc{string}, 
              -- no description yet --
              The \xmlNode{ARMA} node recognizes the following parameters:
                \begin{itemize}
                  \item \xmlAttr{variable}: \xmlDesc{string, optional}, 
                    -- no description yet --
              \end{itemize}

            \item \xmlNode{Function}:\xmlDesc{string}, 
              -- no description yet --
              The \xmlNode{Function} node recognizes the following parameters:
                \begin{itemize}
                  \item \xmlAttr{method}: \xmlDesc{string, optional}, 
                    -- no description yet --
              \end{itemize}

            \item \xmlNode{variable}:\xmlDesc{string}, 
              -- no description yet --

            \item \xmlNode{growth}:\xmlDesc{float}, 
              -- no description yet --
              The \xmlNode{growth} node recognizes the following parameters:
                \begin{itemize}
                  \item \xmlAttr{mode}: \xmlDesc{growthType, optional}, 
                    -- no description yet --
              \end{itemize}
          \end{itemize}

        \item \xmlNode{initial_stored}:
          -- no description yet --

          The \xmlNode{initial_stored} node recognizes the following subnodes:
          \begin{itemize}
            \item \xmlNode{fixed_value}:\xmlDesc{float}, 
              -- no description yet --

            \item \xmlNode{sweep_values}:\xmlDesc{float_list}, 
              -- no description yet --

            \item \xmlNode{opt_bounds}:\xmlDesc{float_list}, 
              -- no description yet --

            \item \xmlNode{ARMA}:\xmlDesc{string}, 
              -- no description yet --
              The \xmlNode{ARMA} node recognizes the following parameters:
                \begin{itemize}
                  \item \xmlAttr{variable}: \xmlDesc{string, optional}, 
                    -- no description yet --
              \end{itemize}

            \item \xmlNode{Function}:\xmlDesc{string}, 
              -- no description yet --
              The \xmlNode{Function} node recognizes the following parameters:
                \begin{itemize}
                  \item \xmlAttr{method}: \xmlDesc{string, optional}, 
                    -- no description yet --
              \end{itemize}

            \item \xmlNode{variable}:\xmlDesc{string}, 
              -- no description yet --

            \item \xmlNode{growth}:\xmlDesc{float}, 
              -- no description yet --
              The \xmlNode{growth} node recognizes the following parameters:
                \begin{itemize}
                  \item \xmlAttr{mode}: \xmlDesc{growthType, optional}, 
                    -- no description yet --
              \end{itemize}
          \end{itemize}
      \end{itemize}

    \item \xmlNode{demands}:
      -- no description yet --
      The \xmlNode{demands} node recognizes the following parameters:
        \begin{itemize}
          \item \xmlAttr{resource}: \xmlDesc{string_list, required}, 
            -- no description yet --
          \item \xmlAttr{dispatch}: \xmlDesc{dispatch_opts, optional}, 
            -- no description yet --
      \end{itemize}

      The \xmlNode{demands} node recognizes the following subnodes:
      \begin{itemize}
        \item \xmlNode{capacity}:
          -- no description yet --
          The \xmlNode{capacity} node recognizes the following parameters:
            \begin{itemize}
              \item \xmlAttr{resource}: \xmlDesc{string, optional}, 
                -- no description yet --
          \end{itemize}

          The \xmlNode{capacity} node recognizes the following subnodes:
          \begin{itemize}
            \item \xmlNode{fixed_value}:\xmlDesc{float}, 
              -- no description yet --

            \item \xmlNode{sweep_values}:\xmlDesc{float_list}, 
              -- no description yet --

            \item \xmlNode{opt_bounds}:\xmlDesc{float_list}, 
              -- no description yet --

            \item \xmlNode{ARMA}:\xmlDesc{string}, 
              -- no description yet --
              The \xmlNode{ARMA} node recognizes the following parameters:
                \begin{itemize}
                  \item \xmlAttr{variable}: \xmlDesc{string, optional}, 
                    -- no description yet --
              \end{itemize}

            \item \xmlNode{Function}:\xmlDesc{string}, 
              -- no description yet --
              The \xmlNode{Function} node recognizes the following parameters:
                \begin{itemize}
                  \item \xmlAttr{method}: \xmlDesc{string, optional}, 
                    -- no description yet --
              \end{itemize}

            \item \xmlNode{variable}:\xmlDesc{string}, 
              -- no description yet --

            \item \xmlNode{growth}:\xmlDesc{float}, 
              -- no description yet --
              The \xmlNode{growth} node recognizes the following parameters:
                \begin{itemize}
                  \item \xmlAttr{mode}: \xmlDesc{growthType, optional}, 
                    -- no description yet --
              \end{itemize}
          \end{itemize}

        \item \xmlNode{minimum}:
          -- no description yet --
          The \xmlNode{minimum} node recognizes the following parameters:
            \begin{itemize}
              \item \xmlAttr{resource}: \xmlDesc{string, optional}, 
                -- no description yet --
          \end{itemize}

          The \xmlNode{minimum} node recognizes the following subnodes:
          \begin{itemize}
            \item \xmlNode{fixed_value}:\xmlDesc{float}, 
              -- no description yet --

            \item \xmlNode{sweep_values}:\xmlDesc{float_list}, 
              -- no description yet --

            \item \xmlNode{opt_bounds}:\xmlDesc{float_list}, 
              -- no description yet --

            \item \xmlNode{ARMA}:\xmlDesc{string}, 
              -- no description yet --
              The \xmlNode{ARMA} node recognizes the following parameters:
                \begin{itemize}
                  \item \xmlAttr{variable}: \xmlDesc{string, optional}, 
                    -- no description yet --
              \end{itemize}

            \item \xmlNode{Function}:\xmlDesc{string}, 
              -- no description yet --
              The \xmlNode{Function} node recognizes the following parameters:
                \begin{itemize}
                  \item \xmlAttr{method}: \xmlDesc{string, optional}, 
                    -- no description yet --
              \end{itemize}

            \item \xmlNode{variable}:\xmlDesc{string}, 
              -- no description yet --

            \item \xmlNode{growth}:\xmlDesc{float}, 
              -- no description yet --
              The \xmlNode{growth} node recognizes the following parameters:
                \begin{itemize}
                  \item \xmlAttr{mode}: \xmlDesc{growthType, optional}, 
                    -- no description yet --
              \end{itemize}
          \end{itemize}

        \item \xmlNode{penalty}:
          -- no description yet --

          The \xmlNode{penalty} node recognizes the following subnodes:
          \begin{itemize}
            \item \xmlNode{fixed_value}:\xmlDesc{float}, 
              -- no description yet --

            \item \xmlNode{sweep_values}:\xmlDesc{float_list}, 
              -- no description yet --

            \item \xmlNode{opt_bounds}:\xmlDesc{float_list}, 
              -- no description yet --

            \item \xmlNode{ARMA}:\xmlDesc{string}, 
              -- no description yet --
              The \xmlNode{ARMA} node recognizes the following parameters:
                \begin{itemize}
                  \item \xmlAttr{variable}: \xmlDesc{string, optional}, 
                    -- no description yet --
              \end{itemize}

            \item \xmlNode{Function}:\xmlDesc{string}, 
              -- no description yet --
              The \xmlNode{Function} node recognizes the following parameters:
                \begin{itemize}
                  \item \xmlAttr{method}: \xmlDesc{string, optional}, 
                    -- no description yet --
              \end{itemize}

            \item \xmlNode{variable}:\xmlDesc{string}, 
              -- no description yet --

            \item \xmlNode{growth}:\xmlDesc{float}, 
              -- no description yet --
              The \xmlNode{growth} node recognizes the following parameters:
                \begin{itemize}
                  \item \xmlAttr{mode}: \xmlDesc{growthType, optional}, 
                    -- no description yet --
              \end{itemize}
          \end{itemize}
      \end{itemize}

    \item \xmlNode{economics}:
      -- no description yet --

      The \xmlNode{economics} node recognizes the following subnodes:
      \begin{itemize}
        \item \xmlNode{lifetime}:\xmlDesc{integer}, 
          -- no description yet --

        \item \xmlNode{CashFlow}:
          -- no description yet --
          The \xmlNode{CashFlow} node recognizes the following parameters:
            \begin{itemize}
              \item \xmlAttr{name}: \xmlDesc{string, required}, 
                -- no description yet --
              \item \xmlAttr{type}: \xmlDesc{string, required}, 
                -- no description yet --
              \item \xmlAttr{taxable}: \xmlDesc{bool, required}, 
                -- no description yet --
              \item \xmlAttr{inflation}: \xmlDesc{string, required}, 
                -- no description yet --
              \item \xmlAttr{mult_target}: \xmlDesc{bool, required}, 
                -- no description yet --
              \item \xmlAttr{period}: \xmlDesc{period_opts, optional}, 
                -- no description yet --
          \end{itemize}

          The \xmlNode{CashFlow} node recognizes the following subnodes:
          \begin{itemize}
            \item \xmlNode{driver}:
              -- no description yet --

              The \xmlNode{driver} node recognizes the following subnodes:
              \begin{itemize}
                \item \xmlNode{fixed_value}:\xmlDesc{float}, 
                  -- no description yet --

                \item \xmlNode{sweep_values}:\xmlDesc{float_list}, 
                  -- no description yet --

                \item \xmlNode{opt_bounds}:\xmlDesc{float_list}, 
                  -- no description yet --

                \item \xmlNode{ARMA}:\xmlDesc{string}, 
                  -- no description yet --
                  The \xmlNode{ARMA} node recognizes the following parameters:
                    \begin{itemize}
                      \item \xmlAttr{variable}: \xmlDesc{string, optional}, 
                        -- no description yet --
                  \end{itemize}

                \item \xmlNode{Function}:\xmlDesc{string}, 
                  -- no description yet --
                  The \xmlNode{Function} node recognizes the following parameters:
                    \begin{itemize}
                      \item \xmlAttr{method}: \xmlDesc{string, optional}, 
                        -- no description yet --
                  \end{itemize}

                \item \xmlNode{variable}:\xmlDesc{string}, 
                  -- no description yet --

                \item \xmlNode{growth}:\xmlDesc{float}, 
                  -- no description yet --
                  The \xmlNode{growth} node recognizes the following parameters:
                    \begin{itemize}
                      \item \xmlAttr{mode}: \xmlDesc{growthType, optional}, 
                        -- no description yet --
                  \end{itemize}
              \end{itemize}

            \item \xmlNode{reference_price}:
              -- no description yet --

              The \xmlNode{reference_price} node recognizes the following subnodes:
              \begin{itemize}
                \item \xmlNode{fixed_value}:\xmlDesc{float}, 
                  -- no description yet --

                \item \xmlNode{sweep_values}:\xmlDesc{float_list}, 
                  -- no description yet --

                \item \xmlNode{opt_bounds}:\xmlDesc{float_list}, 
                  -- no description yet --

                \item \xmlNode{ARMA}:\xmlDesc{string}, 
                  -- no description yet --
                  The \xmlNode{ARMA} node recognizes the following parameters:
                    \begin{itemize}
                      \item \xmlAttr{variable}: \xmlDesc{string, optional}, 
                        -- no description yet --
                  \end{itemize}

                \item \xmlNode{Function}:\xmlDesc{string}, 
                  -- no description yet --
                  The \xmlNode{Function} node recognizes the following parameters:
                    \begin{itemize}
                      \item \xmlAttr{method}: \xmlDesc{string, optional}, 
                        -- no description yet --
                  \end{itemize}

                \item \xmlNode{variable}:\xmlDesc{string}, 
                  -- no description yet --

                \item \xmlNode{growth}:\xmlDesc{float}, 
                  -- no description yet --
                  The \xmlNode{growth} node recognizes the following parameters:
                    \begin{itemize}
                      \item \xmlAttr{mode}: \xmlDesc{growthType, optional}, 
                        -- no description yet --
                  \end{itemize}
              \end{itemize}

            \item \xmlNode{reference_driver}:
              -- no description yet --

              The \xmlNode{reference_driver} node recognizes the following subnodes:
              \begin{itemize}
                \item \xmlNode{fixed_value}:\xmlDesc{float}, 
                  -- no description yet --

                \item \xmlNode{sweep_values}:\xmlDesc{float_list}, 
                  -- no description yet --

                \item \xmlNode{opt_bounds}:\xmlDesc{float_list}, 
                  -- no description yet --

                \item \xmlNode{ARMA}:\xmlDesc{string}, 
                  -- no description yet --
                  The \xmlNode{ARMA} node recognizes the following parameters:
                    \begin{itemize}
                      \item \xmlAttr{variable}: \xmlDesc{string, optional}, 
                        -- no description yet --
                  \end{itemize}

                \item \xmlNode{Function}:\xmlDesc{string}, 
                  -- no description yet --
                  The \xmlNode{Function} node recognizes the following parameters:
                    \begin{itemize}
                      \item \xmlAttr{method}: \xmlDesc{string, optional}, 
                        -- no description yet --
                  \end{itemize}

                \item \xmlNode{variable}:\xmlDesc{string}, 
                  -- no description yet --

                \item \xmlNode{growth}:\xmlDesc{float}, 
                  -- no description yet --
                  The \xmlNode{growth} node recognizes the following parameters:
                    \begin{itemize}
                      \item \xmlAttr{mode}: \xmlDesc{growthType, optional}, 
                        -- no description yet --
                  \end{itemize}
              \end{itemize}

            \item \xmlNode{scaling_factor_x}:
              -- no description yet --

              The \xmlNode{scaling_factor_x} node recognizes the following subnodes:
              \begin{itemize}
                \item \xmlNode{fixed_value}:\xmlDesc{float}, 
                  -- no description yet --

                \item \xmlNode{sweep_values}:\xmlDesc{float_list}, 
                  -- no description yet --

                \item \xmlNode{opt_bounds}:\xmlDesc{float_list}, 
                  -- no description yet --

                \item \xmlNode{ARMA}:\xmlDesc{string}, 
                  -- no description yet --
                  The \xmlNode{ARMA} node recognizes the following parameters:
                    \begin{itemize}
                      \item \xmlAttr{variable}: \xmlDesc{string, optional}, 
                        -- no description yet --
                  \end{itemize}

                \item \xmlNode{Function}:\xmlDesc{string}, 
                  -- no description yet --
                  The \xmlNode{Function} node recognizes the following parameters:
                    \begin{itemize}
                      \item \xmlAttr{method}: \xmlDesc{string, optional}, 
                        -- no description yet --
                  \end{itemize}

                \item \xmlNode{variable}:\xmlDesc{string}, 
                  -- no description yet --

                \item \xmlNode{growth}:\xmlDesc{float}, 
                  -- no description yet --
                  The \xmlNode{growth} node recognizes the following parameters:
                    \begin{itemize}
                      \item \xmlAttr{mode}: \xmlDesc{growthType, optional}, 
                        -- no description yet --
                  \end{itemize}
              \end{itemize}

            \item \xmlNode{depreciate}:\xmlDesc{integer}, 
              -- no description yet --
          \end{itemize}
      \end{itemize}
  \end{itemize}

\clearpage
    \providecommand*{\phantomsection}{}
    \phantomsection
    \addcontentsline{toc}{section}{References}
    \bibliographystyle{ieeetr}
    \bibliography{../HERON_user_manual}
    \end{document}