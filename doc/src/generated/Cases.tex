\section{Cases Introduction}HERON relies on this \xmlNode{xml} node which informs the algorithm as to how the case has to be processed by using the predefined metrics described in the following sections.




\subsection{Case}
  The \xmlNode{Case} contains    the basic parameters needed for a HERON case.

  The \xmlNode{Case} node recognizes the following parameters:
    \begin{itemize}
      \item \xmlAttr{name}: \xmlDesc{string, required}, 
        An appropriate user defined name of the case.
  \end{itemize}

  The \xmlNode{Case} node recognizes the following subnodes:
  \begin{itemize}
    \item \xmlNode{mode}: \xmlDesc{[min, max, sweep]}, 
      Minimize, maximize or sweep over multiple values of capacities.

    \item \xmlNode{metric}: \xmlDesc{[NPV, lcoe]}, 
      This metric can be NPV (Net Present Value) and lcoe (levelized cost of energy) used for
      techno-economic analysis of the power plants.

    \item \xmlNode{differential}: \xmlDesc{[yes, y, true, t, si, vero, dajie, oui, ja, yao, verum, evet, dogru, 1, on, no, n, false, f, nono, falso, nahh, non, nicht, bu, falsus, hayir, yanlis, 0, off, Yes, Y, True, T, Si, Vero, Dajie, Oui, Ja, Yao, Verum, Evet, Dogru, 1, On, No, N, False, F, Nono, Falso, Nahh, Non, Nicht, Bu, Falsus, Hayir, Yanlis, 0, Off]}, 
      Differential represents the additional cashflow generated when building additional capacities.
      This value can be either \xmlString{True} or \xmlString{False}.

    \item \xmlNode{num\_arma\_samples}: \xmlDesc{integer}, 
      Number of copies of the trained signals.

    \item \xmlNode{timestep\_interval}: \xmlDesc{integer}, 
      Time step interval between two values of signal.

    \item \xmlNode{history\_length}: \xmlDesc{integer}, 
      Total length of one realization of the ARMA signal.

    \item \xmlNode{economics}:
      \xmlNode{economics} contains the details of the econometrics     computations to be performed
      by the code.

      The \xmlNode{economics} node recognizes the following subnodes:
      \begin{itemize}
        \item \xmlNode{ProjectTime}: \xmlDesc{float}, 
          Total length of the project.

        \item \xmlNode{DiscountRate}: \xmlDesc{float}, 
          Interest rate required to compute the discounted cashflow (DCF)

        \item \xmlNode{tax}: \xmlDesc{float}, 
          Taxation rate is a metric which represents the      rate at which an individual or
          corporation is taxed.

        \item \xmlNode{inflation}: \xmlDesc{float}, 
          Inflation rate is a metric which represents the     the rate at which the average price
          level of a basket of selected goods and services in an economy increases over some period
          of time.

        \item \xmlNode{verbosity}: \xmlDesc{integer}, 
          Length of the output argument.
      \end{itemize}

    \item \xmlNode{dispatch\_increment}: \xmlDesc{float}, 
      This is the amount of resource to be dispatched in a fixed time interval.
      The \xmlNode{dispatch\\_increment} node recognizes the following parameters:
        \begin{itemize}
          \item \xmlAttr{resource}: \xmlDesc{string, required}, 
            Resource to be consumed or produced.
      \end{itemize}
  \end{itemize}
