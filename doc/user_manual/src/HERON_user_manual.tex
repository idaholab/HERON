% \input{} or \include{} lines will added later
\documentclass[pdf,12pt]{../src/INLreport}
\usepackage{times}
\usepackage{longtable}
\usepackage[FIGBOTCAP,normal,bf,tight]{subfigure}
\usepackage{amsmath}
\usepackage{amssymb}
\usepackage[labelfont=bf]{caption}
\usepackage{pifont}
% \usepackage{enumerate}
\usepackage{listings}
\usepackage{fullpage}
\usepackage{xcolor}          % Using xcolor for more robust color specification
\usepackage{ifthen}          % For simple checking in newcommand blocks
\usepackage{textcomp}

% increase allowable depth of itemize lists
% - note that as of this writing we need a depth of 5.
\usepackage{enumitem}
\setlistdepth{7}
\setlist[itemize]{label=$\circ$}
\renewlist{itemize}{itemize}{7}
% end itemize depth adjustment

% suppress hbox overfull, underfull warnings
\hfuzz=\maxdimen
\hbadness=10000

%\usepackage{authblk}         % For making the author list look prettier
%\renewcommand\Authsep{,~\,}
% Custom colors
\definecolor{deepblue}{rgb}{0,0,0.5}
\definecolor{deepred}{rgb}{0.6,0,0}
\definecolor{deepgreen}{rgb}{0,0.5,0}
\definecolor{forestgreen}{RGB}{34,139,34}
\definecolor{orangered}{RGB}{239,134,64}
\definecolor{darkblue}{rgb}{0.0,0.0,0.6}
\definecolor{gray}{rgb}{0.4,0.4,0.4}
\lstset {
  basicstyle=\ttfamily,
  frame=single
}
\setcounter{secnumdepth}{5}
\lstdefinestyle{XML} {
    language=XML,
    extendedchars=true,
    breaklines=true,
    breakatwhitespace=true,
%    emph={name,dim,interactive,overwrite},
    emphstyle=\color{red},
    basicstyle=\ttfamily,
%    columns=fullflexible,
    commentstyle=\color{gray}\upshape,
    morestring=[b]",
    morecomment=[s]{<?}{?>},
    morecomment=[s][\color{forestgreen}]{<!--}{-->},
    keywordstyle=\color{cyan},
    stringstyle=\ttfamily\color{black},
    tagstyle=\color{darkblue}\bf\ttfamily,
    morekeywords={name,type},
%    morekeywords={name,attribute,source,variables,version,type,release,x,z,y,xlabel,ylabel,how,text,param1,param2,color,label},
}
\lstset{language=python,upquote=true}
\usepackage{titlesec}
\newcommand{\sectionbreak}{\clearpage}
\setcounter{secnumdepth}{4}
\usepackage[utf8]{inputenc}
% Default fixed font does not support bold face
\DeclareFixedFont{\ttb}{T1}{txtt}{bx}{n}{9} % for bold
\DeclareFixedFont{\ttm}{T1}{txtt}{m}{n}{9}  % for normal
\usepackage{listings}
\newcommand\pythonstyle{\lstset{
language=Python,
basicstyle=\ttm,
otherkeywords={self, none, return},             % Add keywords here
keywordstyle=\ttb\color{deepblue},
emph={MyClass,__init__},          % Custom highlighting
emphstyle=\ttb\color{deepred},    % Custom highlighting style
stringstyle=\color{deepgreen},
frame=tb,                         % Any extra options here
showstringspaces=false            %
}}
% Python environment
\lstnewenvironment{python}[1][]
{
\pythonstyle
\lstset{#1}
}
{}
% Python for external files
\newcommand\pythonexternal[2][]{{
\pythonstyle
\lstinputlisting[#1]{#2}}}
\lstnewenvironment{xml}
{}
{}
% Python for inline
\newcommand\pythoninline[1]{{\pythonstyle\lstinline!#1!}}
% Named Colors for the comments below (Attempted to match git symbol colors)
\definecolor{RScolor}{HTML}{8EB361}  % Sonat (adjusted for clarity)
\definecolor{DPMcolor}{HTML}{E28B8D} % Dan
\definecolor{JCcolor}{HTML}{82A8D9}  % Josh (adjusted for clarity)
\definecolor{AAcolor}{HTML}{8D7F44}  % Andrea
\definecolor{CRcolor}{HTML}{AC39CE}  % Cristian
\definecolor{RKcolor}{HTML}{3ECC8D}  % Bob (adjusted for clarity)
\definecolor{DMcolor}{HTML}{276605}  % Diego (adjusted for clarity)
\definecolor{PTcolor}{HTML}{990000}  % Paul
\def\DRAFT{} % Uncomment this if you want to see the notes people have been adding
% Comment command for developers (Should only be used under active development)
\ifdefined\DRAFT
  \newcommand{\nameLabeler}[3]{\textcolor{#2}{[[#1: #3]]}}
\else
  \newcommand{\nameLabeler}[3]{}
\fi
\newcommand{\alfoa}[1] {\nameLabeler{Andrea}{AAcolor}{#1}}
\newcommand{\cristr}[1] {\nameLabeler{Cristian}{CRcolor}{#1}}
\newcommand{\talbpaul}[1] {\nameLabeler{Paul}{PTcolor}{#1}}
% Commands for making the LaTeX a bit more uniform and cleaner
\newcommand{\TODO}[1]    {\textcolor{red}{\textit{(#1)}}}
\newcommand{\xmlAttrRequired}[1] {\textcolor{red}{\textbf{\texttt{#1}}}}
\newcommand{\xmlAttr}[1] {\textcolor{cyan}{\textbf{\texttt{#1}}}}
\newcommand{\xmlNodeRequired}[1] {\textcolor{deepblue}{\textbf{\texttt{<#1>}}}}
\newcommand{\xmlNode}[1] {\textcolor{darkblue}{\textbf{\texttt{<#1>}}}}
\newcommand{\xmlString}[1] {\textcolor{black}{\textbf{\texttt{'#1'}}}}
\newcommand{\xmlDesc}[1] {\textbf{\textit{#1}}} % Maybe a misnomer, but I am
                                                % using this to detail the data
                                                % type and necessity of an XML
                                                % node or attribute,
                                                % xmlDesc = XML description
\newcommand{\default}[1]{~\\*\textit{Default: #1}}
\newcommand{\nb} {\textcolor{deepgreen}{\textbf{~Note:}}~}
\usepackage{bm}
\newcommand{\tensor}[1]{{\bm{#1}}}
\renewcommand{\vec}{\bm}
\newcommand{\unit}[1]{\hat{\bm{#1}}}
\newcommand{\scalarunit}[1]{\hat{#1}}
\usepackage{booktabs}
\usepackage{stmaryrd}
\usepackage{hyperref}
\hypersetup{
    colorlinks,
    citecolor=black,
    filecolor=black,
    linkcolor=black,
    urlcolor=black
}
\newcommand{\wiki}{\href{https://github.com/idaholab/raven/wiki}{RAVEN wiki}}
\usepackage{cite}
\raggedbottom
\setcounter{secnumdepth}{5} % show 5 levels of subsection
\setcounter{tocdepth}{5} % include 5 levels of subsection in table of contents
\title{HERON User Manual}
\author{
 \textbf{\textit{Principal Investigator and Technical Leader:}}
\\Paul W. Talbot\\
\textbf{\textit{Main Developers:}}
\\Paul W. Talbot\\
Dylan J. McDowell\\
R. Daniel Garrett\\
Botros N. Hanna\\
Abhinav Gairola\\
\textbf{\textit{Additional Contributors:}}
\\Jia Zhou\\
Marisol Garrouste\\
}
\date{}
\SANDnum{INL/EXT-20-58976, GDE-939}
\SANDprintDate{\today}
\SANDauthor{Paul W. Talbot, Dylan J. McDowell, R. Daniel Garrett, Botros N. Hanna, Abhinav Gairola, Jia Zhou}
\SANDreleaseType{Revision 1}
\def\component#1{\texttt{#1}}
% ---------------------------------------------------------------------------- %
\newcommand{\systemtau}{\tensor{\tau}_{\!\text{SUPG}}}
% Added by Sonat
\usepackage{placeins}
\usepackage{array}
\newcolumntype{L}[1]{>{\raggedright\let\newline\\\arraybackslash\hspace{0pt}}m{#1}}
\newcolumntype{C}[1]{>{\centering\let\newline\\\arraybackslash\hspace{0pt}}m{#1}}
\newcolumntype{R}[1]{>{\raggedleft\let\newline\\\arraybackslash\hspace{0pt}}m{#1}}
 \begin{document}
    \maketitle
    \SANDmain
    \tableofcontents
    \section{Introduction}
HERON is a generic software plugin of RAVEN to perform stochastic technoeconomic analysis of grid energy-resource systems with economic drivers. The development targets analysis of electricity and
secondary product generation and consumption in regional balancing areas, including flexibility to include arbitrary resources as well as arbitrary resource consumers and producers. HERON is developed to drive optimization via economic drivers such as system cost minimization, profitability, and net present value (NPV) maximization. As a plugin of RAVEN, HERON provides two primary functions: the automatic generation of RAVEN workflows, and models for optimizing high-resolution dispatch of arbitrary systems including resources, resource consumers, and resource producers. HERON leverages the synthetic history training and generation tools, sampling workflows, code Application Programming Interfaces (API), and optimization schemes. 
    \section{ Installation and how to run}
\subsection{Installation}
As a plugin of RAVEN, HERON is installed as a submodule. RAVEN maintains up-to-date
instructions for plugin installation in its manuals and other documentation. As of this writing, plugins
are installed from the command line from within the top level of the RAVEN repository as
\begin{verbatim}
scripts/install_plugins.py ‐s heron
\end{verbatim}
Note that RAVEN must be installed completely to use HERON and its components. At this writing
HERON introduces one additional Python library to the standard set of RAVEN \texttt{Python} dependencies:
\texttt{dill}. The serialization package \texttt{dill} enables information transfer between the HERON templating
algorithm and dispatching algorithm not possible with the standard library set. This library, along with
any future additional libraries required, should be installed correctly via RAVEN’s documented
installation procedure for including plugins.
HERON also requires use of the RAVEN plugin\textt{tCashFlow} , a utility for economic analysis.
As of this writing, \texttt{CashFlow} is an officially supported plugin for RAVEN and is installed standard when plugins are included in RAVEN installation. 
\subsection{How to run}
The code can be run easily on a \texttt{linux/mac} system by defining an alias to the \texttt{Python} script `\texttt{main.py}' in the \texttt{src} folder of  HERON.

    \section{Input Structure}
In the following sections we describe the input structure in a general sense, with details in
following chapters.

\subsection{XML Input}
HERON makes use of the eXtensible Markup Language (XML) for its input structure, similar to RAVEN.
XML is made up of nodes, which have parameters and subnodes. For example:
\begin{lstlisting}[style=XML,morekeywords={class}]
  <node_tag par_name="par_value">
    <sub_tag sub_par_name="sub_par_value">sub_value</sub_tag>
  </node_tag>
\end{lstlisting}
The node's name (or tag) opens the XML element. In the example, we have two nodes, named
\xmlNode{node\_tag} and \xmlNode{sub\_tag}. The node \xmlNode{node\_tag} has a parameter with
name \xmlAttr{par\_name}. The parameter \xmlAttr{par\_name} has the value
\xmlString{par\_value}. Similarly, the \xmlNode{sub\_tag} node has a parameter and
corresponding value. The \xmlNode{sub\_tag} further has a value itself given by
\xmlString{sub\_value}.

We will use the terminology \xmlNode{node}, \xmlNode{subnode}, \xmlAttr{parameter}, and
\xmlString{value} to describe the input structure of HERON.

\subsection{Structure}
The HERON XML Input makes use of three main nodes within the root node \xmlNode{HERON}:
\begin{itemize}
  \item \xmlNode{Case}, in which general features of the desired solve are described, including general
    economics to apply, simulation properties such as project length and time stepping, and so forth.
  \item \xmlNode{Components}, in which the components of the grid system to analyze are defined,
    including their physical processes and economics.
  \item \xmlNode{DataGenerators}, in which data manipulation tools such as synthetic history
  generators and custom code functions.
  \item \xmlNode{TestInfo}, which is reserved for regression tests in HERON, describes the purpose
  of the test and information about the test's implementation.
\end{itemize}

Details for the three nodes used in HERON analyses are enumerated the following sections.
%INSERT_SECTIONS_HERE
	 \section{External Code Integration}
HERON can exchange data with other Integrated Energy Systems (IES) codes to conduct technical or economic analyses for various electricity market structures. More information about integrating HERON with other IES software is here: \url{https://ies.inl.gov/SitePages/FORCE.aspx}

The integration between HERON and other IES tools is still a work in progress. Currently, HERON can communicate with the following codes

\subsection{HYBRID}
The HYBRID repository contains a collection of models representing the physical dynamics of various integrated energy systems and processes. HERON has the capability to load the economic information about the components of the grid system automatically from HYBRID. More information about HYBRID is here: \url{https://github.com/idaholab/HYBRID}

An example that demonstrates this capability can be found at: 
\begin{lstlisting}
/HERON/tests/integration_tests/mechanics/hybrid_load/
\end{lstlisting}

The Python script, that auto-loads the needed economic information from HYBRID to HERON, is named \path{hybrid2heron_economic.py}. This script can be found at:
\begin{lstlisting}
/HERON/src/Hybrid2Heron/
\end{lstlisting}


The \path{hybrid2heron_economic.py} script takes one command-line argument, which is the path of the initial HERON input XML file (or pre-input file) before loading any information from HYBRID

For example, the terminal command looks like this:
\begin{lstlisting}
python hybrid2heron_economic.py pre_heron_input.xml
\end{lstlisting}
A new HERON input XML file, \path{heron_input.xml}, is generated with all the subnodes under the \xmlNode{economics} node loaded from the HYBRID text files. More details about the initial HERON input XML file, the generated (loaded) input XML file and other files at the \path{/HERON/tests/integration_tests/mechanics/hybrid_load/} test are discussed in detail in the following subsections. 

\subsubsection{The initial HERON input XML file}
The initial HERON XML file, \path{pre_heron_input.xml}, structure should be similar to the typical HERON input XML file and must have
\begin{itemize}
\item A \xmlNode{Components} node: 
\item At least one \xmlNode{Component} sub-node under the \xmlNode{Components} node such as:
\begin{lstlisting}[style=XML,morekeywords={class}]
<Component name="component_name"> </Component>
\end{lstlisting}   

\item An empty \xmlNode{economics} node under the \xmlNode{Component} node with the path to the HYBRID text file that includes the needed information as follows:
\begin{lstlisting}[style=XML,morekeywords={class}]
<economics src="path/to/HYBRID/file"> </economics>
\end{lstlisting}   
\end{itemize}
If the \xmlNode{economics} node is not empty, it will remain unaltered when creating the final HERON XML file, \path{heron_input.xml}.

An example of the initial HERON XML file, \path{pre_heron_input.xml}, is located at \path{/HERON/tests/integration_tests/mechanics/hybrid_load/}. 
The \xmlNode{Components} node at the \path{pre_heron_input.xml} looks like this:
\begin{lstlisting}[style=XML,morekeywords={class}]
<Components>
  <Component name="source">
    <!--Other component subnodes-->
    <economics src="Costs/source/source.toml"></economics>
  </Component>
</Components>
\end{lstlisting}
In this example, the economic information of the component \xmlString{source} would be loaded from a text file whose path is \path{/Costs/source/source.toml}. Similarly, for any other component, the economic information can be provided from other text files. 

\subsubsection{The HYBRID text files}
The HYBRID text files are expected to have extensions such as .toml or .txt or .rtf and are located at: 
\path{/HERON/tests/integration_tests/mechanics/hybrid_load/Costs/sink/sink.toml} and 
\path{/HERON/tests/integration_tests/mechanics/hybrid_load/Costs/source/source.toml}. 

The HYBRID files do not have to be in the same folder with the \path{pre_heron_input.xml} as long as the value of the \xmlString{src} parameter at the \xmlNode{economics} node is the path to the corresponding HYBRID text file. The HYBRID text file structure looks like this:
\begin{lstlisting}
Lifetime = 30 #years
VOM = 0 # Just a placeholder to make sure it passes through
Activity = a
\end{lstlisting}

Each line, in the HYBRID text file, includes a variable name and its value plus a comment (if necessary). Any comments must start with the \verb|#| sign. The data should be appropriately auto-loaded from HYBRID text file to HERON XML file even if the text file includes additional irrelevant variables or additional comments at the top or the bottom of the file.

\subsubsection{The generated HERON Input XML file}
The generated input file, \path{heron_input.xml}, includes:
\begin{itemize}
\item All the information that was in the initial HERON input file,       \path{pre_heron_input.xml}
\item All the relevant variables from the HYBRID text files.
\item Default values for additional variables that are found neither at the \path{pre_heron_input.xml} nor at the HYBRID text files but are required by HERON to make sure that the input XML file is complete, and no required nodes or parameters are missing. The comments (warnings) inside the \path{heron_input.xml} tell the user if the values of specific variables are not provided by HYBRID, and if default values are assigned instead. The user should review these comments/warnings.
\end{itemize}

\subsubsection{HYBRID and HERON keywords}
The HYBRID keywords or the HYBRID variables that the \path{hybrid2heron_economic.py} code can identify to create the corresponding HERON nodes are listed in the CSV file, \path{HYBRID_HERON_keywords.csv}, which is located at: \path{/HERON/src/Hybrid2Heron}. Understanding this CSV file is essential, especially if the user plans to add more HYBRID variables or modify them. The CSV file, \path{HYBRID_HERON_keywords.csv} includes the following columns:


\begin{itemize}
    \item \textbf{HYBRID Keyword}: This column lists all the HYBRID variables' names that the Python script, \path{hybrid2heron_economic.py}, can identify. The variables' names in the HYBRID text files must be a subset of the HYBRID variables' names at the \path{HYBRID_HERON_keywords.csv}. Otherwise, the user can either change the variables' names in the \path{HYBRID_HERON_keywords.csv} file or in the HYBRID text files.    
    \item \textbf{Description}: The description or the definition of each HYBRID variable 
    \item \textbf{HERON (Node or Parameter)}: This column specifies if the HYBRID variable is corresponding to a HERON node or a node parameter in the HERON input XML file. \emph{N} refers to a node while \emph{P} refers to a parameter.
    \item \textbf{HERON Node}, \textbf{HERON Subnode} and \textbf{HERON Subsubnode}: These three columns specify the location of the HERON node corresponding to the HYBRID variable. For example, the HYBRID variable \verb|"Activity"| corresponds the sub-sub-node \xmlNode{activity} under the \xmlNode{driver} sub-node under the \xmlNode{CashFlow} node as follows:
    
    \begin{lstlisting}[style=XML,morekeywords={class}]
<CashFlow inflation="none" mult_target="FALSE" name="VOM" taxable="TRUE" type="repeating">
    <driver>
        <activity>a</activity>
    </driver>
</CashFlow>

    \end{lstlisting}
          Also, the HYBRID variable \verb|"VOM_inflation"| corresponds to a \emph{parameter(P)} or an attribute that is called \xmlString{inflation} at the node \xmlNode{CashFlow} under the \xmlNode{economics} node as illustrated in the \xmlNode{economics} node (above).
    
    \item \textbf{Belong to same node}: This column is intended to determine the list of sub-nodes or parameters that belong to the same node. We consider four primary nodes under the \xmlNode{economics} node, which are the component lifetime plus three types of cash flows: The Capital Expenditures (CAPEX) cash flow, the Fixed Operation and Maintenance (FOM) cash flow, and the Variable Operation and Maintenance (VOM) cash flow. These four primary nodes are listed under the \xmlNode{economics} node at the HERON input XML file are as follows:
    
\begin{minipage}{0.93\textwidth}
\begin{lstlisting}[style=XML,morekeywords={class}]
<economics>
  <lifetime>30</lifetime>
  <CashFlow inflation="none" mult_target="FALSE" name="capex" taxable="TRUE" type="one-time"></CashFlow>
  <CashFlow inflation="none" mult_target="FALSE" name="FOM" taxable="TRUE" type="repeating"></CashFlow>
  <CashFlow inflation="none" mult_target="FALSE" name="VOM" taxable="TRUE" type="repeating"></CashFlow>
</economics>
\end{lstlisting}
\end{minipage} \par

The numerical values under the \verb|"Belong to same node"| column are either \verb|0| or \verb|1| or \verb|2| or \verb|3| corresponding to the nodes \xmlNode{lifetime}, \xmlNode{CashFlow name="capex"}, \xmlNode{CashFlow name="FOM"}, \xmlNode{CashFlow name="VOM"} respectively. 

For example, since the HYBRID variable \verb|"capex_inflation"| corresponds to the parameter, \xmlString{inflation}, under the node \xmlNode{CashFlow name="capex"}, the corresponding numerical value under the \verb|"Belong to same node"| is \verb|"1"|. 

Similarly, the \verb|"Amortization_lifetime"| HYBRID variable corresponds to the HERON sub-node \xmlNode{depreciate} under \emph{all} the three cash flow nodes, the corresponding numerical value under the \verb|"Belong to same node"| is \verb|"1,2,3"|. 

Note that we consider three types only of cash flows, but the user can add additional cash flows to the \path{HYBRID_HERON_keywords.csv} file, if needed. 
     
 \item \textbf{Required if HYBRID keyword?}: This column determines if the HYBRID variable needs to be included when building the HERON input XML file even if this HYBRID variable is not provided by the HYBRID text files. 
 
 For example, the HYBRID variable, \verb|"Activity"|, will be included if the \verb|"VOM"| cash flow is present since the node \xmlNode{CashFlow name="VOM"} will be incomplete if the \xmlNode{activity} sub-node is missing. Therefore, for the HYBRID variable, \verb|"Activity"|, the corresponding value under the \verb|"Required if HYBRID keyword?"| column is \verb|"VOM"|
 
 \item \textbf{Default value for the required variable}: This column assigns a default value for any HYBRID variable whose value is not provided by the HYBRID text files if this HYBRID variable is required (see the column \verb|"Required if HYBRID keyword?"|). For example, the default value of the HYBRID variable, \verb|"Activity"|, if required, is \xmlString{electricity} 
\end{itemize}
   %%%%%%%%%%%%%% This is for MOPED/DISPATCHES
   \section{Workflow Options}
HERON has the capability to run different workflows, which expand the flexibility and capabilities of this plugin. Currently, HERON input files control the workflow selection via the \xmlNode{workflow} node, which is a subnode of the \xmlNode{case} node. Each workflow approaches the Techno-economic Analysis (TEA) with a unique problem formulation and solution technique.

\subsection{Default}
The default HERON workflow primarily utilizes RAVEN. In this workflow, RAVEN runs RAVEN to solve a two-level representation of the TEA by utilizing both pyomo and RAVEN's gradient descent optimizer. To run this workflow, insert ``standard`` into the \xmlNode{workflow} node, or simply do not define this node.

\subsection{Monolithic Optimizer for Probabilistic Economic Dispatch (MOPED)}
This workflow formulates the problem as a single-level optimization problem. More specifically, MOPED utilizes TEAL and RAVEN's externalROMloader to generate and solve a pyomo object. To run this workflow, insert ``MOPED`` into the \xmlNode{workflow} node.
\noindent MOPED provides an alternate approach to solving the TEA provided by the input file. The solutions MOPED and the default workflow provide should be similar in standard cases.

\subsubsection{Motivations}
The primary motivations and potential benefits of MOPED include:
\begin{itemize}
    \item \textbf{Computational time:} In cases where the IES in question is following a cooperative dispatch heuristic (The standard dipatcher for the default workflow applies here), the single level formulation maintains the advantage of utilizing a more deterministic optimization algorithm
          (`ipopt`') than gradient search. This results from the gradient descent treating the NPV cost function as a black box with capacities as input and NPV as output. In constrast, MOPED's pyomo object has an algebraic expression generated with TEAL, allowing for more direct application of optimization techniques.
    \item \textbf{Seeding for more complicated scenarios:} In future versions of HERON, FARM will be available as a validation tool for HERON. FARM introduces new constraints that limit aspects of dispatch, such as ramping and setpoints, to ensure physical feasibility of the system's operation.
          For this use of HERON, MOPED could provide an initial solution input to FARM. This may reduce the number of iterations required to meet the validation criteria of the analysis.
    \item \textbf{Validation of default workflow/Confirmation of bilevel-monolithic equivalence:} Comparing the results between these two workflows provides a litmus test for the validity of either.
\end{itemize}

\subsubsection{Limitations}
MOPED is limited to the TEA's where the dispatch and capacity selection agents are cooperative. In other words, MOPED cannot solve analyses where maximizing dispatch value reduces the total NPV value. Possible scenarios include deregulated markets, direct competition, and agent-based dispatch.

Additionally, MOPED has limitations in terms of acceptable inputs, which currently include:
\begin{itemize}
    \item Custom functions for prices, VRE capacities, demand, etc.
    \item Components that do not start operation at project start
    \item Components with (component life x rebuild count $<$ project life)
    \item Components that produce, consume, or demand multiple resources
    \item Components with multiple cashflows of a same type
\end{itemize}
Development is focussed on reducing the number of items on this list. The end goal of MOPED is to maintain the same capabilities as the default workflow.

\subsection{DISPATCHES}
The DISPATCHES workflow builds a monolithic pyomo object with cost functions from TEAL, which is similar to MOPED. In constrast, this workflow utilizes IDAES modeling to include physics within the pyomo model.


   %%%%%%%%%%%%%% STILL NEEDS TO BE MADE
    \providecommand*{\phantomsection}{}
    \phantomsection
    \addcontentsline{toc}{section}{References}
    \bibliographystyle{ieeetr}
    \bibliography{../../HERON_user_manual}
    \end{document}
