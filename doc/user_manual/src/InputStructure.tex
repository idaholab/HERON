\section{Input Structure}
In the following sections we describe the input structure in a general sense, with details in
following chapters.

\subsection{XML Input}
HERON makes use of the eXtensible Markup Language (XML) for its input structure, similar to RAVEN.
XML is made up of nodes, which have parameters and subnodes. For example:
\begin{lstlisting}[style=XML,morekeywords={class}]
  <node_tag par_name="par_value">
    <sub_tag sub_par_name="sub_par_value">sub_value</sub_tag>
  </node_tag>
\end{lstlisting}
The node's name (or tag) opens the XML element. In the example, we have two nodes, named
\xmlNode{node\_tag} and \xmlNode{sub\_tag}. The node \xmlNode{node\_tag} has a parameter with
name \xmlAttr{par\_name}. The parameter \xmlAttr{par\_name} has the value
\xmlString{par\_value}. Similarly, the \xmlNode{sub\_tag} node has a parameter and
corresponding value. The \xmlNode{sub\_tag} further has a value itself given by
\xmlString{sub\_value}.

We will use the terminology \xmlNode{node}, \xmlNode{subnode}, \xmlAttr{parameter}, and
\xmlString{value} to describe the input structure of HERON.

\subsection{Structure}
The HERON XML Input makes use of three main nodes within the root node \xmlNode{HERON}:
\begin{itemize}
  \item \xmlNode{Case}, in which general features of the desired solve are described, including general
    economics to apply, simulation properties such as project length and time stepping, and so forth.
  \item \xmlNode{Components}, in which the components of the grid system to analyze are defined,
    including their physical processes and economics.
  \item \xmlNode{DataGenerators}, in which data manipulation tools such as synthetic history
  generators and custom code functions.
  \item \xmlNode{TestInfo}, which is reserved for regression tests in HERON, describes the purpose
  of the test and information about the test's implementation.
\end{itemize}

Details for the three nodes used in HERON analyses are enumerated the following sections.